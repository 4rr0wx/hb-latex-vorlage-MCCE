\usepackage[utf8]{inputenc}
%ENtfernt rote ränder von Hyperlinks
\usepackage[hidelinks]{hyperref}
%Wird nicht mehr benötigt damit Umlaute im PDF richtig angezeigt werden
%\usepackage[T1]{fontenc}	

%Wird fürs Anzeigen von Grafiken benötigt
\usepackage{graphicx}

%Wird benötigt damit Tabellenbeschriftung auf Deutsch
\usepackage[ngerman]{babel} 

% verwendet für subfigure Umgebungen
\usepackage{subcaption}

% verwendet für todo notes (\todo)
\setlength {\marginparwidth }{2cm}
\usepackage[colorinlistoftodos]{todonotes}

%Package für Listing (code listing - Umgebung "lstlisting")
\usepackage{listings}

%Packages für Quellenverzeichnis
% Package für \url Befehl
\usepackage{url}
% Package-Abhängigkeit für \appto Befehl (wird für den nächsten Kommando (UrlBreaks) benötigt)
\usepackage{etoolbox}

%wird für autoref benötigt, andererseits erstellt es im pdf klickbare Referenzen
\usepackage{hyperref}
\usepackage{amsmath}
%wird für cref benötigt
\usepackage[ngerman]{cleveref}

\usepackage{relsize}
\usepackage[ngerman]{babel}
\usepackage{gensymb} 
\usepackage{textcomp}

\usepackage[sfdefault]{arimo}
% Font encoding 
\usepackage[T1]{fontenc}  
% This package allows the user to specify the input encoding 
\usepackage[utf8]{inputenc}
\usepackage{setspace}

%eventuell benötigte Pakete

 % Das Paket wrapfig ermöglicht es von Schrift umflossene Bilder und Tabellen
% \usepackage{wrapfig}

% ermöglicht das anpassen der list Umgebungen (itemize, enumerate, description)
% \usepackage{enumitem}

% bessere Möglichkeiten in die Platzierung von Abbildungen, etc. einzugreifen
% \usepackage{float}

% Abstand zwischen zwei Absätzen kontrollieren 
% \usepackage{parskip}

% konfiguriert Parameter des hyperref packages, so dass Linkfarben angepasst werden - muss die parameterlose Variante des Befehls (\usepackage{hyperref}) ersetzen
% \usepackage[colorlinks=true, urlcolor=blue, linkcolor=black]{hyperref}

\usepackage[acronym]{glossaries}
\usepackage{blindtext}
\usepackage{tocloft}
\usepackage{lastpage}
